\subsection{Efficiency}



%According to \cite{roomzilla3}, productivity should not be measured by the time spent at the work place or at a desk. Instead measures based on actual results should be preferred. 

According to \cite{roomzille3}, one of the biggest issues in companies are technical problems with presentations during meetings. Some of the problems identified by the authors include sub-optimal setup, used (but not booked) rooms, and technical failures like failing connections between computer and presenting device. They propose streamlined processes to improve the setup of meetings and make them more efficient. They also propose that meetings rooms can lead to better and more efficient meetings. Examples for improvements include simpler tech setups and room booking software.

In \cite{hbcommunications}, the authors mention how AI and machine learning could be used to save employee time. They describe a smart call system with an automated menu that learns to direct calls to the correct department. This will reduce the time that is spent on redirecting callers and therefore improve the employees' efficiency. 

In \cite{roomzilla9} the authors estimate, how much cost managing rooms by hand can cause. The authors assume, that an office manager who is tasked with room bookings spends 90 minutes per day to answer emails, instant messages, etc. to manage room bookings. The average salary of an office manager in the US is \(20,65\text{USD}\). Based on this, the authors of \cite{roomzilla9} estimate the costs of this task around \(681,45\text{USD}\) per month. The authors also mention some problems that can cause hidden costs: both late running meetings and overbooked rooms prevent employees from using their time for actual work. 

A similar result is mentioned by \cite{iotagenda}. The authors of this article describe how tracking room usage can lead to a better use of time. They propose a system that tracks room usage and makes the data available via Outlook or an appropriate alternative. According to \cite{iotagenda}, the availability of room information leads to less distractions and waste of time. 


\cite{roomzilla3} mentions that an exciting workplaces may lead to more creative employees and more comraderie between employees. Typically it is assumed that such a playful environment is detrimental to the productiviy of employees. However, according to \cite{metrooffice}, such environments can actually lead to a more successful business because the employees are more motivated and creative.

In \cite{hbcommunications} it is stated that a streamlined communication and improved connectivity leads to better and faster collaboration. The authors describe how this can make finding experts in the organization easier.


\cite{hbcommunications} outlines how a smart office can be used to conserve energy. Smart lights and devices can be programmed to turn on when employees need them. On another note, these lights and devices can be switched off when nobody needs them.

Speaking of communication, having a smart workplace can have a big impact on overall corporate culture within an organization. People expect their workplaces to have technology that’s at least as good as what they have at home—preferably better—and this is especially true for those digital natives, the Millennials. Studies show that happy employees are productive employees and productive employees lead to prosperous businesses. Equally as important is the ability of companies to not only attract, but also to retain top talent. When you create a workplace that is a pleasure to be in and work in and provide state-of-the-art technology, it goes a long way toward making employees want to stay around. easier life for office management: lights turn on and off automatically, temperature regulation via thermostats \(\rightarrow\) safe energy. more fun workplace with: coffee pots, fitness equipment, or wearables that track fitness and reward employees for
reaching certain fitness goals\cite{hbcommunications}
