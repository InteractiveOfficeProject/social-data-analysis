\documentclass[journal,10pt]{IEEEtran}
\usepackage{hyperref}
\usepackage{todonotes}
\usepackage{amssymb}
\usepackage{amsmath}
\usepackage{mathtools}
\usepackage{graphicx}

\title{Social Data Analysis}
\author{Jan Lippert \(<\)\href{mailt:ljan@mail.upb.de}{ljan@mail.upb.de}\(>\)}
\date{\today}

\newcommand{\subtask}[1]{\begin{quote}\textbf{#1}\end{quote}}

\begin{document}
\begin{abstract}
Task: People commonly share their experiences online.  For you, this can turn into a gold mine of valuable insights.  Interested in improving the restaurant going experience?  Study a few dozen restaurant reviews and you will learn a lot about what makes or breaks a great outing.  Want to design for citizen scientists?  Study a few blogs on http://citizenscientistsleague.com.  Blogs, reviews, forums, Twitter, and other social media are full of personal stories of people interacting with products, attempting to accomplish interesting tasks, etc.  You cannot ask a question of a blog post, but you can read lots of them.  Social media can productively complement your in-person data gathering by letting you observe a much larger number of people in a more diverse set of situations.   Analyze stories/posts/etc. from at least a dozen different individuals from whichever collection of sources that you find most appropriate for your project.

Note: again, make sure that your data gathering online is done in an ethical manner: do not attempt to obtain information under false pretenses (e.g., do not pretend to be a young parent to get onto a GardenMoms forum) and do not re-share information that people disclosed with a reasonable expectation of confidentiality.  Of course, you do not need to use a consent form to analyze information that people disclosed in a public setting.

\begin{itemize}
  \item Did you analyze social media data from at least a dozen people?  Have you submitted the list of sources?
  \item Did you identify at least five needs, problems, or pain points?
  \item Are all of your needs, problems, or pain points clearly rooted in evidence?  Did you explain what observations led to each of the insights? Did you include quotes/excerpts when necessary?
\end{itemize}
\end{abstract}

\maketitle

\section{Introduction}
\section{Something more}

\begin{quote}
There's lot of activity in the office\\
Tracking the activity via IoT can lead into new insights about these behaviours\\
Knowledge can be used to improve: productivity, engagement, safety\\
E.g.: Customize lighting and heating \(\rightarrow\) more comfortable working enviroment \\
Tracking Room usage via sensors, connect to outlook etc\(\rightarrow\) Dynamic room booking, usage trends\\
\(\Rightarrow\) less distractions, more focus on satisftaction and efficiency
\cite[iotagenda]
\end{quote}

\begin{quote}
is an organization that is driven by its mission and vision \\
builds flexible, mobile and dynamic teams – and supports and rewards their collaboration \\
develops leaders and connectors, not task managers \\
is agile and knows when to adopt new ways of working \\
understands technology and builds systems for collaborative communication around a network of teams.
 -- \cite[trina]
\end{quote}

\begin{quote}
well-organized, efficient,
enjoyable workplace and serve the employees who inhabit that space on a daily basis.\\
impacts everything from energy efficiency and convenience (like lights that automatically go on when a room is occupied), to enhanced
collaboration spaces, to improved employee communication.\\
Speaking of communication, having a smart workplace can have a big impact on overall
corporate culture within an organization. People expect their workplaces to have technology
that’s at least as good as what they have at home—preferably better—and this is especially
true for those digital natives, the Millennials. Studies show that happy employees are
productive employees and productive employees lead to prosperous businesses. Equally as
important is the ability of companies to not only attract, but also to retain top talent. When you
create a workplace that is a pleasure to be in and work in and provide state-of-the-art
technology, it goes a long way toward making employees want to stay around.\\
easier life for office management: lights turn on and off automatically, temperature regulation via thermostats \(\rightarrow\) safe energy\\
more fun workplace with: coffee pots, fitness equipment, or wearables that track fitness and reward employees for
reaching certain fitness goals\\
~\\
Utilizing AI:\\
LEarning systems can help the company to save time and energy\\
e.g. smart call system with automated menu learns to automatically direct calls to the correct department or person\(\rightarrow\) less time wsted on redirection\\
e.g. worker has problem with (potentially internal) application, AI system detects application and routes the support call to correct IT team\\
remote support is possible via audio and video conferences, pc management via remote access\\
still in infant stage but great potential\\
~\\
better communitaction\\
streamlining communication and improving connectivity leads to better and faster collaboration\\
connecting to the ``right guy'' gets easier\\
 -- \cite[hbcommunications]
\end{quote}

\begin{quote}
 -- \cite[]
\end{quote}

\bibliographystyle{ieeetr}  
\bibliography{sda}

\end{document}
