\documentclass[journal,10pt]{IEEEtran}
\usepackage{hyperref}
\usepackage{todonotes}
\usepackage{amssymb}
\usepackage{amsmath}
\usepackage{mathtools}
\usepackage{graphicx}

\title{Social Data Analysis}
\author{Jan Lippert \(<\)\href{mailt:ljan@mail.upb.de}{ljan@mail.upb.de}\(>\)}
\date{\today}

\begin{document}
\begin{abstract}
Task: People commonly share their experiences online.  For you, this can turn into a gold mine of valuable insights.  Interested in improving the restaurant going experience?  Study a few dozen restaurant reviews and you will learn a lot about what makes or breaks a great outing.  Want to design for citizen scientists?  Study a few blogs on http://citizenscientistsleague.com.  Blogs, reviews, forums, Twitter, and other social media are full of personal stories of people interacting with products, attempting to accomplish interesting tasks, etc.  You cannot ask a question of a blog post, but you can read lots of them.  Social media can productively complement your in-person data gathering by letting you observe a much larger number of people in a more diverse set of situations.   Analyze stories/posts/etc. from at least a dozen different individuals from whichever collection of sources that you find most appropriate for your project.

Note: again, make sure that your data gathering online is done in an ethical manner: do not attempt to obtain information under false pretenses (e.g., do not pretend to be a young parent to get onto a GardenMoms forum) and do not re-share information that people disclosed with a reasonable expectation of confidentiality.  Of course, you do not need to use a consent form to analyze information that people disclosed in a public setting.

\begin{itemize}
  \item Did you analyze social media data from at least a dozen people?  Have you submitted the list of sources?
  \item Did you identify at least five needs, problems, or pain points?
  \item Are all of your needs, problems, or pain points clearly rooted in evidence?  Did you explain what observations led to each of the insights? Did you include quotes/excerpts when necessary?
\end{itemize}
\end{abstract}

\maketitle

\section{Introduction}
\section{Something more}

\begin{quote}
There's lot of activity in the office\\
Tracking the activity via IoT can lead into new insights about these behaviours\\
Knowledge can be used to improve: productivity, engagement, safety\\
E.g.: Customize lighting and heating \(\rightarrow\) more comfortable working enviroment \\
Tracking Room usage via sensors, connect to outlook etc\(\rightarrow\) Dynamic room booking, usage trends\\
\(\Rightarrow\) less distractions, more focus on satisftaction and efficiency
\cite[iotagenda]
\end{quote}

\begin{quote}
is an organization that is driven by its mission and vision \\
builds flexible, mobile and dynamic teams – and supports and rewards their collaboration \\
develops leaders and connectors, not task managers \\
is agile and knows when to adopt new ways of working \\
understands technology and builds systems for collaborative communication around a network of teams.
 -- \cite[trina]
\end{quote}

\begin{quote}
well-organized, efficient,
enjoyable workplace and serve the employees who inhabit that space on a daily basis.\\
impacts everything from energy efficiency and convenience (like lights that automatically go on when a room is occupied), to enhanced
collaboration spaces, to improved employee communication.\\
Speaking of communication, having a smart workplace can have a big impact on overall
corporate culture within an organization. People expect their workplaces to have technology
that’s at least as good as what they have at home—preferably better—and this is especially
true for those digital natives, the Millennials. Studies show that happy employees are
productive employees and productive employees lead to prosperous businesses. Equally as
important is the ability of companies to not only attract, but also to retain top talent. When you
create a workplace that is a pleasure to be in and work in and provide state-of-the-art
technology, it goes a long way toward making employees want to stay around.\\
easier life for office management: lights turn on and off automatically, temperature regulation via thermostats \(\rightarrow\) safe energy\\
more fun workplace with: coffee pots, fitness equipment, or wearables that track fitness and reward employees for
reaching certain fitness goals\\
~\\
Utilizing AI:\\
LEarning systems can help the company to save time and energy\\
e.g. smart call system with automated menu learns to automatically direct calls to the correct department or person\(\rightarrow\) less time wsted on redirection\\
e.g. worker has problem with (potentially internal) application, AI system detects application and routes the support call to correct IT team\\
remote support is possible via audio and video conferences, pc management via remote access\\
still in infant stage but great potential\\
~\\
better communitaction\\
streamlining communication and improving connectivity leads to better and faster collaboration\\
connecting to the ``right guy'' gets easier\\
 -- \cite[hbcommunications]
\end{quote}

\begin{quote}
Solid foundations needed: secure and reliable wifi. Used for working and IoT infrastructure. unified communication is another issue. 52\% of businesses want UC.\\
Improve facility management? Get rid of manual controls and automate things. Space usage patterns enable companies to make better use of space etc. and save energy, e.g. by turning off lighting and power for unused rooms and places. \\
Underside desk sensors can be used but don'T utilize ict completely. Wifi tracking does. Can be used to track movements across the office, Ping LAN ports to find out if port is used  \(\rightarrow\) usage patterns. \\
Bottleneck IT Department? USe self-management of services. You need to balance ease of use and security, but the result can free up the precious time of the IT guys.\\
ORcheastrating an IoT platform is really important. All devices must be managed. Good platform support different IoT devices regardless of vendor. 
 -- \cite[scmedia]
\end{quote}

\begin{quote}
industrial workplaces can be dangerous, wearables may contribute to safer environments. apps and mobile devices can help workers in hazardous environents by: sending notifications, visual recongition, and communication with machines. \\
example: jan 2012, one worker died when investigating noise in an oxygen furnance. pipe burst because of built-up pressure, causing death. the accident could have been avoided by utilizing ``smart devices'': machine sensors connected to smart-glasses or watches \(\rightarrow\) tell workers to stay away, pipe pressure data analysis \(\rightarrow\) investigation of possible cause without getting close\\
Greatest challenge is SW, not HW. SW must be intuitive to be used easily, SW must be connected to the IT and the devices. Improved data solutions for streamed machine data is also needed\\
Needed: policy and protocols for data governance, privacy, data administration of personal health-information, and extensive redcordings of mic and video feeds. All of these problems are related to SW.
 -- \cite[sda-wired]
\end{quote}

\begin{quote}
Don't measure productivity by time spent at work place or at desk. MEasure actual work output. Fun and exciting workplaces lead to more creative employees and more comraderie between employees. This in turn leads to a more successful business. \\
biggest issue in many companies: problems with presentations during meetings. problems: tangled wires, used but unbooked rooms, failing connections to the devices. Streamlined processes improve usability and efficiency, improved meetings rooms (wireless, better booking, LM booking) also lead to better and more efficient meetings, incl .scheduling meetings.\\ 
 -- \cite[roomzilla3]
\end{quote}

\begin{quote}
to return roi of room booking software: how much time does the office manager spend to schedule rooms? how much time do the employees lose because of double bookings and lack of available rooms?\\
Based on US numbers (salary for office manager 20,65 USD/h). If OM spends 90m a day for managing room bookings, this amounts to 681,45 USD per month to manage just rooms. \\
most common issues when not using room management software: late running meetings and overbooked rooms. can lead to a waste of time and also money.  
 -- \cite[roomzilla9]
\end{quote}

\begin{quote}
4 key drivers or enablers of change within the workplace strategy area: 1. Cost/price pressure; 2. Sustainability \& Corporate Responsibility; 3. Technology; and 4. the War for talent \& Productivity area\\
1. rents are very expensive, especially in nordic countries. utilizization of office space rarely exceeds 50\%.\\
2. sustainability becomes more and more important. sustainability becomes more of a business opportunity\\
3. technology becomes more and more sophisticated and can enable change. \\
4. talented workers are scarce and productivity needs to be optimized. different issues here: unemployment is high but both public and private employers struggle to find people\todo{drop?}, variation at workplace (multi-generation workplace, multi-cultural environment), disengaged workers doing subpar work, workplaces hindering productivity, strong relationship between comfort and self-reported productivity (diff of 25\% between comfortable and uncomfortable staff)\\
people are very expensive, improving their wellbeing can highly impact the financial state of the company. 
 -- \cite[hub13]
\end{quote}

\begin{quote}
major factor in office design: supporting different styles of work. collaborative work needs places where ``the atmosphere is
conducive to innovation'', i.e. not as boring as a normal meeting room. however, there must also be place to receive phone calls and to work concentrated on a project etc., i.e. silent working areas are needed.\\
If these areas are provided, the employees must be able to freely move between these areas. therefore the office must provide an``instant, real-time picture of what's going on'' so that the people see where space is available. It is also crucial that the office provides a way to see where colleagues are working as they are able to freely move between different areas.\\
Example of working together: spontaneous meetings, just see who'S around, where's room and go. 
 -- \cite[tieto]
\end{quote}

\begin{quote}
living lab is a real open space working area which includes an experimentation room. This lab was created because studies in real working environments are hard to do: the situation cannot be controlled and the study must impact the working people as less as possible. To conduct research, controlled study bedingungen\todo{translate} are needed. However, results achieved in especially created experiment rooms are hard to transfer to real world situations.

The living lab combines these conflicting areas of interest. The lab itself is an open space with with flexible working stations. The lab also has a clean desk policy, i.e. the people working there do not have a fixed desk but choose their working place each day according to availability.

The living lab has researched different types of technologies. Examples include electrochromatic glass which can be turned dark on demand to prevent sunlight from shining trough or personlaized air flow optimization for workers. The living lab tests these technologies in simulations and real life situations. Another area of research is light and acoustic optimization. Good lighting and acoustics cannot be measured directly as most people only notice bad lighting or bad acoustics. 

Modern LEDs provide great ways to improve the worker's productivity by adapting intensity and the color specturm. It has been shown in recent research that the color spectrum directly influences the activity and the biorhythmus of people. In acoustics, a good level of ``noise'' has to be found. Most open area offices are too quiet and as such talks between colleaguas and phone calls distract other people. However, too loud environments are also detrimental to work. Therefore the right balance as to be fouind to optimize the worker's productivity.   
 -- \cite[living-lab]
}
\end{quote}

\bibliographystyle{ieeetr}  
\bibliography{sda}

\end{document}
