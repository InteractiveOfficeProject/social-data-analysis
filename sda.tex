\documentclass[journal,10pt]{IEEEtran}
\usepackage{hyperref}
%\usepackage{todonotes}
\usepackage{amssymb}
\usepackage{amsmath}
\usepackage{mathtools}
\usepackage{graphicx}

\newcommand{\todo}[1]{\textbf{#1}}

\title{Social Data Analysis}
\author{Jan Lippert \(<\)\href{mailt:ljan@mail.upb.de}{ljan@mail.upb.de}\(>\)}
\date{\today}

\begin{document}
\begin{abstract}
Task: People commonly share their experiences online.  For you, this can turn into a gold mine of valuable insights.  Interested in improving the restaurant going experience?  Study a few dozen restaurant reviews and you will learn a lot about what makes or breaks a great outing.  Want to design for citizen scientists?  Study a few blogs on http://citizenscientistsleague.com.  Blogs, reviews, forums, Twitter, and other social media are full of personal stories of people interacting with products, attempting to accomplish interesting tasks, etc.  You cannot ask a question of a blog post, but you can read lots of them.  Social media can productively complement your in-person data gathering by letting you observe a much larger number of people in a more diverse set of situations.   Analyze stories/posts/etc. from at least a dozen different individuals from whichever collection of sources that you find most appropriate for your project.

Note: again, make sure that your data gathering online is done in an ethical manner: do not attempt to obtain information under false pretenses (e.g., do not pretend to be a young parent to get onto a GardenMoms forum) and do not re-share information that people disclosed with a reasonable expectation of confidentiality.  Of course, you do not need to use a consent form to analyze information that people disclosed in a public setting.

\begin{itemize}
  \item Did you analyze social media data from at least a dozen people?  Have you submitted the list of sources?
  \item Did you identify at least five needs, problems, or pain points?
  \item Are all of your needs, problems, or pain points clearly rooted in evidence?  Did you explain what observations led to each of the insights? Did you include quotes/excerpts when necessary?
\end{itemize}
\end{abstract}

\maketitle

\section{Introduction}
There's lot of activity in the office. Tracking the activity via IoT can lead into new insights about these behaviours. Knowledge can be used to improve: productivity, engagement, safety \cite{iotagenda}

well-organized, efficient, enjoyable workplace and serve the employees who inhabit that space on a daily basis.\cite{hbcommunications}

impacts everything from energy efficiency and convenience (like lights that automatically go on when a room is occupied), to enhanced
collaboration spaces, to improved employee communication. \cite{hbcommunications}

\subsection{Related Research}
Studies in real working environments are hard to do: the situation cannot be controlled and the study must impact the working people as less as possible. To conduct research, a controlled environment is needed. Researchers at the university of Kaiserslautern created the living lab to merge these requirements. The living lab is an open space office that was designed with the goal to conduct research in the area of Smart Offices \cite{living-lab}.

% The lab has a clean desk policy, i.e. the people working there do not have a fixed desk but choose their working place each day according to availability \cite{living-lab}.

The living lab has researched different types of technologies. Examples include electrochromatic glass which can be turned dark on demand to prevent sunlight from shining trough or personlaized air flow optimization for workers. The living lab tests these technologies in simulations and real life situations. Another area of research is light and acoustic optimization. Good lighting and acoustics cannot be measured directly as most people only notice bad lighting or bad acoustics \cite{living-lab}.




\section{Areas of Interest}\todo{better title}

\subsection{Sustainability}

\cite{hbcommunications} outlines how a smart office can be used to conserve energy. Smart lights and devices can be programmed to turn on when employees need them. On another note, these lights and devices can be switched off when nobody needs them.

Another possibility is the automatic regulation of room temperature based on the time of day and other factors. The importance of temperature is also outlined in \cite{living-lab}. The authors mention that employees expect good thermic regulation in the office and that it is necessary to focus on work. If the temperature is not right, employees get distracted 	 
\subsection{Well-Being}

Modern LEDs provide great ways to improve the worker's productivity by adapting intensity and the color spectrum. It has been shown in recent research that the color spectrum directly influences the activity and biorhythms of people  \cite{living-lab}. \cite{iotagenda} also highlights, how smart lighting can be used to create a more comfortable working environment.

Another important factor of well-being is acoustics. Most open area offices are too quiet and as such talks between colleagues and phone calls distract other people. However, too loud environments are also detrimental to work. Therefore the right balance as to be found \cite{living-lab}. 

Another possibility of the smart and interactive office is the automatic regulation of room temperature based on the time of day. Both \cite{iotagenda} and \cite{living-lab} outline the importance of temperature in the well-being of employees. People expect good thermal regulation in the office and it is also necessary to focus on work. But not everybody does feel temperature the same way. The living lab therefore developed and currently a ``climatic chair'' that helps each individual to regulate his or her working surrounding temperature \cite{living-lab}.



\subsection{Safety}
Industrial workplaces like factories can be dangerous. While this is not directly related to a smart or interactive office, it is nonetheless an important factor in a smart workplace. \cite{sda-wired} lists a deadly accident which could possible have been prevented by utilizing modern technology.

In January 2012, one worker of the ArcelorMittal Burns Harbor steel-mill died while investigating noise in an oxygen furnace. The cause of death was a bursting pipe that released hot steam. The burst was caused by previously built-up pressure. A smart workplace could have prevented this accident by tracking pressure data in the pipe and warning workers to keep clear of the dangerous area.

Possible implementations of such a system could use apps, mobile devices, and wearables. In case of danger, acoustic and visual notifications could be send to the user. While such devices are widely available, \cite{sda-wired} mentions that software is lacking behind. The software of such systems must be intuitive to use. Also, the whole office and workplace has to be integrated: IT, machines, sensors, and finally the workers' devices. Since sensors produce a lot of data, \cite{sda-wired} also mentions that improved algorithms for streamed data analysis are needed.

% Needed: policy and protocols for data governance, privacy, data administration of personal health-information, and extensive redcordings of mic and video feeds. All of these problems are related to SW. \cite{sda-wired}


\subsection{Efficiency}



%According to \cite{roomzilla3}, productivity should not be measured by the time spent at the work place or at a desk. Instead measures based on actual results should be preferred. 

According to \cite{roomzilla3}, one of the biggest issues in companies are technical problems with presentations during meetings. Some of the problems identified by the authors include sub-optimal setup, used (but not booked) rooms, and technical failures like failing connections between computer and presenting device. They propose streamlined processes to improve the setup of meetings and make them more efficient. They also propose that meetings rooms can lead to better and more efficient meetings. Examples for improvements include simpler tech setups and room booking software.

In \cite{hbcommunications}, the authors mention how AI and machine learning could be used to save employee time. They describe a smart call system with an automated menu that learns to direct calls to the correct department. This will reduce the time that is spent on redirecting callers and therefore improve the employees' efficiency. 

In \cite{roomzilla9} the authors estimate, how much cost managing rooms by hand can cause. The authors assume, that an office manager who is tasked with room bookings spends 90 minutes per day to answer emails, instant messages, etc. to manage room bookings. The average salary of an office manager in the US is \(20,65\text{USD}\). Based on this, the authors of \cite{roomzilla9} estimate the costs of this task around \(681,45\text{USD}\) per month. The authors also mention some problems that can cause hidden costs: both late running meetings and overbooked rooms prevent employees from using their time for actual work. 

A similar result is mentioned by \cite{iotagenda}. The authors of this article describe how tracking room usage can lead to a better use of time. They propose a system that tracks room usage and makes the data available via Outlook or an appropriate alternative. According to \cite{iotagenda}, the availability of room information leads to less distractions and waste of time. 


\cite{roomzilla3} mentions that an exciting workplaces may lead to more creative employees and more comraderie between employees. Typically it is assumed that such a playful environment is detrimental to the productiviy of employees. However, according to \cite{metrooffice}, such environments can actually lead to a more successful business because the employees are more motivated and creative.

In \cite{hbcommunications} it is stated that a streamlined communication and improved connectivity leads to better and faster collaboration. The authors describe how this can make finding experts in the organization easier.


\cite{hbcommunications} outlines how a smart office can be used to conserve energy. Smart lights and devices can be programmed to turn on when employees need them. On another note, these lights and devices can be switched off when nobody needs them.

Speaking of communication, having a smart workplace can have a big impact on overall corporate culture within an organization. People expect their workplaces to have technology that’s at least as good as what they have at home—preferably better—and this is especially true for those digital natives, the Millennials. Studies show that happy employees are productive employees and productive employees lead to prosperous businesses. Equally as important is the ability of companies to not only attract, but also to retain top talent. When you create a workplace that is a pleasure to be in and work in and provide state-of-the-art technology, it goes a long way toward making employees want to stay around. easier life for office management: lights turn on and off automatically, temperature regulation via thermostats \(\rightarrow\) safe energy. more fun workplace with: coffee pots, fitness equipment, or wearables that track fitness and reward employees for
reaching certain fitness goals\cite{hbcommunications}


\todo{paperless office}
\subsection{Collaboration}\todo{work on this part}
major factor in office design: supporting different styles of work. collaborative work needs places where ``the atmosphere is
conducive to innovation'', i.e. not as boring as a normal meeting room. however, there must also be place to receive phone calls and to work concentrated on a project etc., i.e. silent working areas are needed.\cite{tieto}

If these areas are provided, the employees must be able to freely move between these areas. therefore the office must provide an``instant, real-time picture of what's going on'' so that the people see where space is available. It is also crucial that the office provides a way to see where colleagues are working as they are able to freely move between different areas. \cite{tieto}

Example of working together: spontaneous meetings, just see who'S around, where's room and go. \cite{tieto}

\cite{roomzilla3} mentions that an exciting workplaces may lead to more creative employees and more comraderie between employees. Typically it is assumed that such a playful environment is detrimental to the productiviy of employees. However, according to \cite{metroffice}, such environments can actually lead to a more successful business because the employees are more motivated and creative.

In \cite{hbcommunications} it is stated that a streamlined communication and improved connectivity leads to better and faster collaboration. The authors describe how this can make finding experts in the organization easier.

Speaking of communication, having a smart workplace can have a big impact on overall corporate culture within an organization. People expect their workplaces to have technology that’s at least as good as what they have at home—preferably better—and this is especially true for those digital natives, the Millennials. Studies show that happy employees are productive employees and productive employees lead to prosperous businesses. Equally as important is the ability of companies to not only attract, but also to retain top talent. When you create a workplace that is a pleasure to be in and work in and provide state-of-the-art technology, it goes a long way toward making employees want to stay around. easier life for office management: lights turn on and off automatically, temperature regulation via thermostats \(\rightarrow\) safe energy. more fun workplace with: coffee pots, fitness equipment, or wearables that track fitness and reward employees for
reaching certain fitness goals\cite{hbcommunications}




\subsection{Cost saving}
to return roi of room booking software: how much time does the office manager spend to schedule rooms? how much time do the employees lose because of double bookings and lack of available rooms? \cite[roomzilla9]

Based on US numbers (salary for office manager 20,65 USD/h). If OM spends 90m a day for managing room bookings, this amounts to 681,45 USD per month to manage just rooms. \cite[roomzilla9]

most common issues when not using room management software: late running meetings and overbooked rooms. can lead to a waste of time and also money.  \cite[roomzilla9]
\section{Conclusion}
\paragraph{smart office useful?}
\paragraph{identified needs}
collaboration, efficiency, saving costs. great factor: well-being, adapt office to worker


% 
\begin{quote}
Solid foundations needed: secure and reliable wifi. Used for working and IoT infrastructure. unified communication is another issue. 52\% of businesses want UC.\cite[scmedia]

Improve facility management? Get rid of manual controls and automate things. Space usage patterns enable companies to make better use of space etc. and save energy, e.g. by turning off lighting and power for unused rooms and places. \cite[scmedia]

Underside desk sensors can be used but don'T utilize ict completely. Wifi tracking does. Can be used to track movements across the office, Ping LAN ports to find out if port is used  \(\rightarrow\) usage patterns. \\
Bottleneck IT Department? USe self-management of services. You need to balance ease of use and security, but the result can free up the precious time of the IT guys. \cite[scmedia]

ORcheastrating an IoT platform is really important. All devices must be managed. Good platform support different IoT devices regardless of vendor. \cite[scmedia]
\end{quote}

\bibliographystyle{ieeetr}  
\bibliography{sda}

\end{document}


