\section{Introduction}\todo{work on this part}
There's lot of activity in the office. Tracking the activity via IoT can lead into new insights about these behaviours. Knowledge can be used to improve: productivity, engagement, safety \cite{iotagenda}

well-organized, efficient, enjoyable workplace and serve the employees who inhabit that space on a daily basis.\cite{hbcommunications}

impacts everything from energy efficiency and convenience (like lights that automatically go on when a room is occupied), to enhanced
collaboration spaces, to improved employee communication. \cite{hbcommunications}

\subsection{Related Research}
Studies in real working environments are hard to do: the situation cannot be controlled and the study must impact the working people as less as possible. To conduct research, a controlled environment is needed. Researchers at the university of Kaiserslautern created the living lab to merge these requirements. The living lab is an open space office that was designed with the goal to conduct research in the area of Smart Offices \cite{living-lab}.

% The lab has a clean desk policy, i.e. the people working there do not have a fixed desk but choose their working place each day according to availability \cite{living-lab}.

The living lab has researched different types of technologies. Examples include electrochromatic glass which can be turned dark on demand to prevent sunlight from shining trough or personlaized air flow optimization for workers. The living lab tests these technologies in simulations and real life situations. Another area of research is light and acoustic optimization. Good lighting and acoustics cannot be measured directly as most people only notice bad lighting or bad acoustics \cite{living-lab}.


