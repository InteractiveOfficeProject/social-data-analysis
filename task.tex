\begin{abstract}
Task: People commonly share their experiences online.  For you, this can turn into a gold mine of valuable insights.  Interested in improving the restaurant going experience?  Study a few dozen restaurant reviews and you will learn a lot about what makes or breaks a great outing.  Want to design for citizen scientists?  Study a few blogs on http://citizenscientistsleague.com.  Blogs, reviews, forums, Twitter, and other social media are full of personal stories of people interacting with products, attempting to accomplish interesting tasks, etc.  You cannot ask a question of a blog post, but you can read lots of them.  Social media can productively complement your in-person data gathering by letting you observe a much larger number of people in a more diverse set of situations.   Analyze stories/posts/etc. from at least a dozen different individuals from whichever collection of sources that you find most appropriate for your project.

Note: again, make sure that your data gathering online is done in an ethical manner: do not attempt to obtain information under false pretenses (e.g., do not pretend to be a young parent to get onto a GardenMoms forum) and do not re-share information that people disclosed with a reasonable expectation of confidentiality.  Of course, you do not need to use a consent form to analyze information that people disclosed in a public setting.

\begin{itemize}
  \item Did you analyze social media data from at least a dozen people?  Have you submitted the list of sources?
  \item Did you identify at least five needs, problems, or pain points?
  \item Are all of your needs, problems, or pain points clearly rooted in evidence?  Did you explain what observations led to each of the insights? Did you include quotes/excerpts when necessary?
\end{itemize}
\end{abstract}